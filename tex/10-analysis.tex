\chapter{Аналитический раздел}
\label{cha:analysis}


В данном разделе анализируется и классифицируется существующая всячина и рассматриваются способы создания новой всячины.

\section{Анализ того и сего}

В работе можно использовать разделы \verb|\chapter|, \verb|\section|, подразделы \verb|\subsection|, подподраздела \verb|\subsubsection| и параграфа \verb|\paragraph|. Однако, следует помнить, что разделы разрешены только в основной части работы. Во введении, заключении и других структурных элементах они запрещены. Старайтесь избегать случаев, когда в разделе существует только один подраздел.

\subsection{Пример использования подраздела}
\subsubsection{Пример использования подподраздела}

\paragraph{Проверка} параграфа. Вроде работает.
\paragraph{Вторая проверка} параграфа. Опять работает.

\begin{itemize}
\item Это список с палочками. 
\item Хотя он соответствует требованиям ГОСТ, лучше использовать список с номерами, чтобы облегчить чтение.
\end{itemize}

\subsection{Пример использования блок-схемы}

\subsubsection*{Заголовки разделов}

В работе можно использовать и подподпункты, например \verb|\subsubsection|. Однако, лучше не нумеровать этот уровень разделов, чтобы не усложнять структуру документа.

\section{Подсистема всякой ерунды}

Существует несколько подсистем, среди которых можно выделить:

\begin{enumerate}
\item Перечисление с номерами.
\item Нумерация первого уровня. В соответствии с ГОСТ, первый уровень нумеруется буквами, а на втором уровне "--- цифрами. Элементы на первом уровне выравниваются как обычные абзацы. Проверим теперь вложенные списки.
\begin{enumerate}
\item Нумерация второго уровня.
\item Нумерация второго уровня. Проверяем на очень длинной строке, чтобы убедиться, что выравнивание работает правильно.
\end{enumerate}
\item По мнению Лукьяненко, человеческий мозг старается свести любую проблему к выбору из трех вариантов.
\item Четвертый (и последний) элемент списка.
\end{enumerate}
