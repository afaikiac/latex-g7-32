\chapter{Технологический раздел}
\label{cha:impl}

В данном разделе описано изготовление и требование всячины. Кстати,
в Latex нужно эскейпить подчёркивание (писать <<\verb|some\_function|>> для \Code{some\_function}).

\ifPDFTeX
% TODO: сделать рабочее решение для листингов c русским текстом в pdflatex
\else

Для вставки кода есть пакет \texttt{minted}. Он хорош всем кроме: необходимости Python (есть во всех нормальных (нет, Windows, я не про тебя) ОС) и Pygments и того, что нормально работает лишь в \XeLaTeX.

\ifdefined\NoMinted
Но к сожалению, у вас, по-видимому, не установлен Python или pygmentize.
\else
Можно пользоваться расширенным BFN:

\begin{listing}[H]
\begin{ebnfcode}
 letter = "A" | "B" | "C" | "D" | "E" | "F" | "G"
       | "H" | "I" | "J" | "K" | "L" | "M" | "N"
       | "O" | "P" | "Q" | "R" | "S" | "T" | "U"
       | "V" | "W" | "X" | "Y" | "Z" ;
digit = "0" | "1" | "2" | "3" | "4" | "5" | "6" | "7" | "8" | "9" ;
symbol = "[" | "]" | "{" | "}" | "(" | ")" | "<" | ">"
       | "'" | '"' | "=" | "|" | "." | "," | ";" ;
character = letter | digit | symbol | "_" ;
 
identifier = letter , { letter | digit | "_" } ;
terminal = "'" , character , { character } , "'" 
         | '"' , character , { character } , '"' ;
 
lhs = identifier ;
rhs = identifier
     | terminal
     | "[" , rhs , "]"
     | "{" , rhs , "}"
     | "(" , rhs , ")"
     | rhs , "|" , rhs
     | rhs , "," , rhs ;
 
rule = lhs , "=" , rhs , ";" ;
grammar = { rule } ;
\end{ebnfcode}
\caption{EBNF определённый через EBNF}
\label{lst:ebnf}
\end{listing}

А вот в листинге \ref{lst:c} на языке C работают русские комменты. Спасибо Pygments и Minted за это.

\begin{listing}[H]
\cfile{res/code/test.c}
\caption{Пример — test.c} 
\end{listing}
\label{lst:c}

\fi
\fi
